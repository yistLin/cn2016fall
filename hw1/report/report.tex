% report.tex
% a report of the homework 1 - Computer Network 2016 Fall

\documentclass[14pt,a4paper]{extarticle}
\usepackage[margin=2.54cm]{geometry}
\usepackage{xeCJK}
\usepackage{fancyhdr}
\usepackage{lastpage}

\setCJKmainfont{cwTeXFangSong}

\usepackage{color}
\usepackage{listings}
\lstset{ %
language=Python,                % choose the language of the code
basicstyle=\footnotesize,       % the size of the fonts that are used for the code
numbers=left,                   % where to put the line-numbers
numberstyle=\footnotesize,      % the size of the fonts that are used for the line-numbers
stepnumber=1,                   % the step between two line-numbers. If it is 1 each line will be numbered
numbersep=5pt,                  % how far the line-numbers are from the code
backgroundcolor=\color{white},  % choose the background color. You must add \usepackage{color}
showspaces=false,               % show spaces adding particular underscores
showstringspaces=false,         % underline spaces within strings
showtabs=false,                 % show tabs within strings adding particular underscores
frame=single,           % adds a frame around the code
tabsize=2,          % sets default tabsize to 2 spaces
captionpos=b,           % sets the caption-position to bottom
breaklines=true,        % sets automatic line breaking
breakatwhitespace=false,    % sets if automatic breaks should only happen at whitespace
escapeinside={\%*}{*)}          % if you want to add a comment within your code
}

\title{Computer Network\\ Homework 1 Report}
\author{林義聖\\ B03902048}
\date{\today}

\pagestyle{fancy}
\fancyhf{}
\lhead{\small{\jobname.pdf}}
\rhead{\small{Computer Network 2016 Fall}}
\cfoot{\small{\thepage\ of \pageref{LastPage}}}

\begin{document}

\maketitle
\thispagestyle{fancy}

\section*{Introduction}

Our goal is to create an \textbf{IRC Robot}. The programming language I chose to do this job is \textit{Python}. There are some needed files to run the program:
\begin{itemize}
	\item \texttt{main} --- a executable bash script
	\item \texttt{config} --- config for the IRC robot
	\item \texttt{robot.py} --- the program of robot
\end{itemize}
My robot is called \textit{yeast\_robot}, and I've implemented the following functions:
\begin{itemize}
	\item \textit{@help} --- print out help message
	\item \textit{@repeat} --- repeat sentence
	\item \textit{@cal} --- simple calculator implemented by myself
	\item \textit{@play} --- play guess number game with robot
	\item \textit{@guess} --- guess number
	\item \textit{@weather} --- get 3-days weather forecast of the city in Taiwan
\end{itemize}

\section*{Program Structure}

\begin{lstlisting}
# import dependencies
import requests, re
...
# initial settings
server_host = 'irc.freenode.net'
...
# useful functions
def setup_channel():
...
# connect to server
sock.connect((server_host, server_port))
...
# send setup message: NICK, JOIN, ...
send_msg('JOIN ' + channel_name + ' ' + channel_key + '\r\n')
...
# infinite loop to receive and process message
while True:
	data = sock.recv(4096)
	if data:
		recv_msg = data.decode('utf-8').strip('\r\n')

		# lots of regular expression inside the loop
		# server is checking the robot
    matched = re.match(r'^PING :', recv_msg)
    if matched is not None:
			...
		# someone is speaking
    matched = re.match(r'^:(.+)!(.+) PRIVMSG #(\w+) :(.+)$', recv_msg)
    if matched is not None:
    	...
    	# someone call @help
    	match_help = re.match(r'^@help$', message)
      if match_help is not None:
      	...
      # another command handler
      ...
\end{lstlisting}

\section*{Challenge \& Solution}

It's a little bit hard at first to get all the stuff I need to join an IRC channel. I've tried so many times to join correctly, but my robot was still kicked out by the channel. Finally, I found that I should setup my username for authentication by sending ''\texttt{USER ... : Hi.}'' before trying to join a channel.

\section*{Reflection}

I think this homework assignment is interesting and a little bit challenging. Creating a chatroom robot is a quite useful skill!

\section*{How to Excute}

This program is written in \textit{Python}, so you must have \texttt{python3} installed in your environment. You can just type \texttt{./main} to run the program. The executable file \textit{main} is written as a bash script in order to run \textit{robot.py} with \textit{Python 3}.

\end{document}
